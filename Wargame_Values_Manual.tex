\documentclass{article}
%\usepackage{fullpage}
\usepackage{hyperref}

\begin{document}

\title{Wargame: Red Dragon Mechanics Manual}
\date{\today}
\author{Resident Mario}
\maketitle

\newpage
\tableofcontents
\newpage

\section{Movement}
\subsection{Speed}

A unit's speed is its off-road speed: the speed at which the unit moves whenever
it is ordered to move, but not to move fast. For units that can't use roads,
and therefore can't move fast---infantry, planes, helicopters, and naval
units---this is the only speed at which they can move.

Speed is lowest for infantry units---obviously, as they move on foot. Infantry
can only manage between 15 and 30 kph.

Vehicular units can be broadly divided into two types by their system of
locomotion: tracked vehicles and wheeled ones. Wheeled vehicles tend to be much
faster, but also much lighter overall. Tracked vehicles can do anywhere between
35 and 80 kph (with the exception of the M-18, which manages 100 kph); wheeled
ones, between 50 and 110 kph.

Helicopters move at speeds of between 180 and 350 kph. Since they have to
accelerate at takeoff and decelerate on arrival, it takes them a few seconds to
reach top speed or to stop.

Airplanes move at constant speeds of between 500 and 1100 kph.

Naval units move at speeds measured in knots---each kt is equal to approximately
1.85 kph. They move at speeds of 14 to 32 kt, or about 26 to 59 kph. Like
helicopters they take a little bit of time to accelerate or to stop, though
significantly less. Quite unrealistically, they are also all capable of
turning in place.

\begin{center}
    \begin{tabular}{ | l | p{9cm} | }
    \hline
    Speed (KPH) & Infantry Units \\ \hline
    15 & ATGM teams, MANPADS teams, some Specialist Infantry.  \\ 
    20 & Line infantry squads. \\ 
    25 & Fast/Shock infantry squads. \\ 
    30 & Very fast/Commando infantry squads. \\
    \hline
    \end{tabular}
\end{center}

\begin{center}
    \begin{tabular}{ | l | p{9cm} | }
    \hline
    Speed (KPH) & Vehicular Units \\ \hline
    35 & I-Hawk pattern tracked anti-air units. \\ 
    40--45 & Especially slow tracked units, such as the T-62M. \\
    50 & Transport trucks.  \\ 
    50--80 & Most tanks. \\
    100 & South Korean M-18, fastest tracked unit.\\
    80--100 & Fast wheeled units. \\ 
    110 & The fastest ground unit, the P4 Milan 2 ATGM jeep. \\ \hline
    \end{tabular}
\end{center}

\begin{center}
    \begin{tabular}{ | l | p{9cm} | }
    \hline
    Speed (KPH) & Helicopter Units \\ \hline
    180 & Mi-4s. \\ 
    200 & Mi-2s. \\ 
    205--240 & Bell UH-1 Hueys and other light NATO transports. \\
    250 & Mi-8s, SA 330 Pumas. \\ 
    280--310 & NATO fast and heavy transports, and gunships. \\
    330 & Mi-24s. \\ 
    350 & AH-1W SuperCobra. \\ \hline

    \end{tabular}
\end{center}

\begin{center}
    \begin{tabular}{ | l | p{9cm} | }
    \hline
    Speed (KPH) & Planes \\ \hline
    500 & A-10 Thunderbolt II.\\
    600 & Slow planes---mostly cheap bombers. \\ 
    750 & Mid-to-low tier bombers and ground-attack aircraft. \\
    900 & MiG-29s, MiG-23s, F-4 Phantoms, and others. \\
    1000 & Default. The most common value. \\
    1100 & MiG-31 and MiG-31M Foxbat interceptors. \\
    \hline
    \end{tabular}
\end{center}

\subsection{Road Speed}

Wheeled and tracked vehicular units, touched upon earlier, are distinguished in
large part by their differing road speeds---the speeds they achieve when fast moving
down a road. Tracked units all have a road speed of 110 kph, regardless of their
off-road speed; similarly, wheeled units all have a road speed of 150 kph,
regardless of their off-road capacity.

No other units are capable of fast-moving down roads, and so no other units have
road speeds.

% \begin{center}
%     \begin{tabular}{ | l | p{9cm} | }
%     \hline
%     Type & Road Speed \\ \hline
%     Tracked & 110 KPH. \\
%     Wheeled & 150 KPH. \\
%     \hline
%     \end{tabular}
% \end{center}

\subsection{Amphibious Speed}

Vehicular units both tracked and wheeled with this trait can move onto and off
of rivers and other bodies of water. This statistic ranges from N/A, in which
case the unit cannot entreat onto rivers at all, up to 53 kph. It's generally
the unit's off-road speed, rounded up.

\begin{center}
    \begin{tabular}{ | l | p{9cm} | }
    \hline
    Speed (KPH) & Vehicular Units \\ \hline
    N/A & All non-marine units, which cannot enter water at all. \\
    18--33 kph & Most cheap water-traversing transports.\\
    40--48 kph & Most other water-traversing units.\\
    50 kph & Most fast water-traversing units, including the LAV.\\
    53 kph & The TPz Fuchs and TPz Fuchs Milan (FRG).\\
    \hline
    \end{tabular}
\end{center}

\subsection{Sailing Type}

Present only for ships and controls which sorts of waters they can enter. All
ships are capable of traversing deep water, but are weeded out below that. Deep
Sea ships can only traverse deep sea; coasters can approach the coastline; and
riverine boats can go down rivers, though they still cannot pass under most
bridges.

% \begin{center}
%     \begin{tabular}{ | l | p{9cm} | }
%     \hline
%     Type & Vehicular Units \\ \hline
%     River & Riverine boats (cannot pass under most bridges, however). \\
%     Coaster & Coastal boats, which can sail on the light blue sections of water,
%     right up to the shoreline. \\
%     Deep Sea & Blue-water boats which can only occupy deep (dark blue) waters.\\
%     \hline
%     \end{tabular}
% \end{center}

\subsection{Forest Movement}

Infantry move just as fast in forest as they do on lighter terrain, but
vehicular units that move through forest are significantly slowed
down---tracked units move at 0.50 times their off-road speed, and wheeled one at
0.33 times their off-road speed. This means that maximum speed when moving
through forest ranges from 17.5 to 40 kph for tracked units (50 kph for the
M-18), and 16.5 to 36.3 for wheeled ones.

\subsection{Turn Radius}

Airplanes cannot turn in place like all other units (including, suspiciously,
ships) can, and instead have a ``turn radius'' that determines, in meters, the
radius of their turns. This is an important statistic for ground attack aircraft
(it allows them to turn away from a target instead of evacuating after an
attack, which helps maintain distance from enemy air superiority fighters) and
particularly for air superiority fighters. Turn radii range from 150 to 500
meters.

\subsection{Autonomy}

How far a unit can move before it runs out of gas and is not longer capable of
moving, in kilometers. Ranges between 200 and 700 km for tanks; between 150 and
1200 km for other vehicular units; and between 300 and 2000 for helicopters.
Infantry, naturally, have infinite autonomy, as do, more surpisingly, supply
units and naval units.

Though naturally helicopters should be guzzling fuel whenever they are airbourn,
in Wargame they only use up fuel when moving. Hovering does not use up any fuel,
nor does landing or changing altitude. Thus if a helicopter runs out of fuel
while on the move, it will stop and land in place, not crash into the ground.

\begin{center}
    \begin{tabular}{ | l | p{9cm} | }
    \hline
    Range (KM) & Vehicular Units \\ \hline
    150 KM & LVKV fm/43 visual SPAAG (SWE). \\
    200-480 KM & Old, heavy units. \\
    500-700 KM & Most units. \\
    700+ KM & High range transports and jeeps. \\
    1000 KM & Very high range transports and jeeps. \\
    1200 KM & Stolly wheeled transport (ANZAC). \\
    \hline
    \end{tabular}
\end{center}

\subsection{Time over Target}

Airplanes do not have an autonomy, but rather a time over target, which ranges
from 60 to 105 seconds. This statistic is of secondary importance, but does come
in handy at times on air superiority fighters. Planes that exceed their autonomy
run out of fuel and have to leave the battlefield, automatically going into
``Evac Bingo''.

Since airplanes always fly at the same speed, their actual autonomy is their
speed multiplied by their autonomy (though for planes, time over target is a
more interesting statistic). The largest autonomy in the game belongs to the
Su-27PU, which covers slightly under 28 km on a tank of fuel.
The lowest autonomy belongs to the A-10 Thunderbolt II---ironically because it
has both the lowest possible speed and highest possible autonomy---which can
only do just over 14.5 km before running out of fuel.

\section{Vision}

\subsection{Stealth}

Stealth effects the visibility of your unit to other units. The Stealth options
are ``Poor'', ``Medium'', ``Good'', ``Very Good'', and ``Exceptional''.

\begin{center}
    \begin{tabular}{ | l | l | }
    \hline
    Stealth & Units \\ \hline
    Poor & All non-recon ground vehicles, most other units.  \\ 
    Medium & All recon scout vehicles, some stealthy units.  \\
    Good & Basic infantry. \\
    Very Good & Recon infantry squads. \\
    Exceptional & Scout snipers, the F-117 Nighthawk.\\
    \hline
    \end{tabular}
\end{center}

Stealth effects how well the unit is able to hide from other units, interacting
with optics and terrain in this regard. All unit types have stealthiness in
their domains, including ships (one example, the La Fayette) and planes (the
F-117 Nighthawk in particular).

\subsection{Optics}

Regular units have either ``Poor'' or ``Medium'' optics. Units with ``Good'' or
better optics are considered recon units, and ``Very Good'' is the baseline for
recon. Units can have either optics or air detection; units with air detection
have ``Poor'' optics, and vice versa.

\begin{center}
    \begin{tabular}{ | l | l | }
    \hline
    Optics & Units \\ \hline
    Poor & Most vehicles.\\
    Medium & Some higher-costed combat vehicles, infantry.\\
    Good & Cheap recon units.\\
    Very Good & Most recon units.\\
    Exceptional & High-end recon units.\\
    \hline
    \end{tabular}
\end{center}

Coastal and riverine ships also have optics, of between ``Good'' and
``Exceptional'' quality.

A unit's optics tells you how well and out to what distance your unit can see
the enemy. Forest or hedgerows offer cover to the unit, and make it
significantly harder to keep it spotted; ubran sectors make infantry, in most
cases, all but invisible to units outside of the town, unless they are firing
their weapon. Weapon firing noise increases spotting distance, but by what
amount is not known, though easily testable.

The maximum optical spotting distance, by a unit with exceptional optics
against a no-stealth unit on an open field, is 5000 m. The minimum spotting
distance, by a regular infantry unit intra-town against an exceptionally
stealthy occupier, is 450 m. The most important values to know are 3500 m, the
open-field spotting distance of a ``Very Good'' recon unit, and 2150 m, the
effectiveness of the same against forest or hedgelines.

For a breakdown of how Optics interacts with Stealth: 
\url{https://www.dropbox.com/s/l2u8w7tuj7igiul/WargameRD_Hidden_Knowledge_Spreadsheet.xls}.

\subsection{Air Detection}

Units which are targetted at the air domain---airplanes and anti-air
units---have no optics, but air detection instead.

\begin{center}
    \begin{tabular}{ | l | l | }
    \hline
    Air Detection & Units \\ \hline
    Medium & The B-5---which, we will see, breaks a lot of rules.\\
    Good & Low-cost anti-air units, most low-cost planes.\\
    Very Good & Most anti-air units, most planes.\\
    Exceptional & High-end air superiority fighters.\\
    \hline
    \end{tabular}
\end{center}

Air detection lengths are not currently know. Note that ``Good'' optics is
anything but for airplanes---the opposing aircraft can get to within
medium AAM firing range before it is spotted with these optics. And the B-5 is
just blind, really.

Since airplanes fly at roughly the same altitude relative to one another, their
spotting radius given any particular level of optics is higher than that of
ground-based anti-air units, which have to reach up to the aircraft's altitude.

\subsection{Optics Sea}

Naval optics exist seperately from regular optics, but naval units can be
sighted using regular optics as well: the difference is that units with naval
optics are able to spot ships out much further, as this optics
type is grossly more effective on the water (and does nothing on land).
Nonetheless, recon units can be used to scout ships (all but one lack any
stealth, and there's no cover on the water, after all).

\begin{center}
    \begin{tabular}{ | l | l | }
    \hline
    Optics Sea & Units \\ \hline
    Good & Lynx HA S.2 ASM helicopter.\\
    Very Good & Some ships, most ASM helicopters.\\
    Exceptional & Most capital ships, some ASM helicopters.\\
    \hline
    \end{tabular}
\end{center}

\subsection{Weapon Firing Noise}

Some weapons are ``noiser'' then others, and this is simulated in the engine
with regards to weapons firing. Most weapons, when fired, greatly increase the
ease with which a unit can be spotted, but some do more than others---
sometimes units can fire their weapons without being spotted at all. A few
``silenced'' weapons, carried by infantry, are completely silent, and make the
unit no easier to spot at all.

Units firing non-silenced weapons are 1 to 8 times as easy to spot. This makes
the maximum spotting distance in the game, going off of the values in the previous
section, 40000 m, or 40 km. It's rather unlikely, however, that nothing will
block the view of spotter at these ranges, at least on the ground.

\section{Damage Resistance}

\subsection{Strength}

All units in Wargame have a certain set amount of health or ``strength'', ranging
from 2 to 15 for most units, and between 40 and 300 for naval ones. Infantry
units range the most in health for standard units, coming in squads anywhere
between two-man sniper teams to fifteen-man heavy infantry and militia squads;
ten is the standard amount across all categories, and is particularly
standardized among air units and tanks, which emph{almost} never veer from that
count.

\begin{center}
    \begin{tabular}{ | l | p{7.5cm} | }
    \hline
    Strength & Infantry Units \\ \hline
    2 & Scout snipers, ATGM teams, MANPADS teams.  \\ 
    5 & Light scouts, Specialist infantry squads. \\ 
    10 & Line infantry squads. \\
    15 & Heavy and Militia infantry squads. \\ \hline
    \end{tabular}
\end{center}

\begin{center}
    \begin{tabular}{ | l | p{7.5cm} | }
    \hline
    Strength & Helicopter Units \\ \hline
    4 & Utility helicopters. \\
    6 & Light helicopters. \\ 
    8 & Medium helicopters. \\ 
    10 & Heavy helicopters. \\ \hline
    \end{tabular}
\end{center}

\begin{center}
    \begin{tabular}{ | l | p{7.5cm} | }
    \hline
    Strength & Vehicular Units \\ \hline
    5 & Light units, particularly jeeps. \\
    10 & Everything else. \\ \hline
    \end{tabular}
\end{center}

A unit's health is displayed below its name when selected. You can also tell at
a glance when a unit is at low health by a ``repair'' icon that appears above it
in-game. Though the use of two to ten blocks for health seems to imply that unit
health can only be a whole number, this is not true---the game engine
calculates health as a float with 32 units of accuracy, and then rounds up or
down when appropriate to give you the displayed health.

The recently added B-5 is the singular exception to this common design point,
having 15 health.

\subsection{Armor}

Armor is applied to the four combat sides of the unit: front, sides, back, and
top. Most of the armor in the game comes in the form of tank armor, which ranges
from 2/2/1/1 on the most anachronistic surplus-issue cavalry tanks (the Chinese
ZTS-63-1 in particular) to 23/11/6/4 on the most up-armored superheavy tank (the
Challenger 2).

Combat-capable vehicles tend to have at least 1 armor on the front, sides, and
top, and almost always at least one armor in the back, though combat systems
mounted on jeeps and trucks usually do not. Tracked units tend to have heavier
armor then wheeled ones when serving similar roles, which is balanced against
their lower speeds. Wheeled units have a very low cealing to how high their
armor can go, with the heaviest units carrying merely 2/2/1/1 armor.

Although all units have the capacity to have armor in the game engine,
not all of them use it---infantry can never have armor. Only two series of
planes have (light) armor, the A-10 Thunderbolt II and the Su-25 Frogfoot
series (up to 2/1/2/1 on the Su-25T), making them unique amongst planes. Most
heavy attack helicopters have a little bit of armor as well in the front and
the sides, with the notable exception of the AH-1 Cobra series---the most
heavily armored helicopter, the Ka-50/Ka-52 Akula, has 1/1/1/0 armor.

\begin{center}
    \begin{tabular}{ | l | l | l | }
    \hline
    Max Armor & Unit Class & Units \\ \hline
    1/1/1/0 & Helicopters & Ka-50/Ka-52 Akula \\
    1/1/1/1 & Supply unit & MTP-LB \\
    2/2/1/1 & Wheeled units & Various \\
    2/1/2/1 & Airplane & Su-25T \\
    7/3/2/2 & Infantry carrier & BTR-T \\
    15/6/3/2 & Anti-air guns & Challenger Marksman \\
    20/9/3/3 & Control vehicles & T-80UK \\
    20/12/6/3 & Support vehicles & Chimera \\
    23/11/6/4 & Tanks & Challenger 2 \\ \hline
    \end{tabular}
\end{center}

Units with zero armor on a side vulnerable to attack can be attacked by and
take damage from any weapon all the way down to infantry small arms. Units with
1 armor where attacked are considered ``bulletproofed'' and take very little
damage from small arms fire, 0.1 damage per unit of HE. Units with more than 1
armor ignore damage from small arms fire completely, and can only be killed by
infantry when hit with their anti-tank launchers. A high enough armor value will
in some situations prevent a unit from taking any damage at all!

\section{Range}

\subsection{Range -- Ground}

Range against ground is the range of the weapon against targets on the ground
(this includes landed helicopters). Most units are able to attack in this
domain even if only in terms of self-defense, with the notable exception of most
missile anti-air systems. Ranges go from 455 m for infantry submachine guns to
42 km (effectively the entire map) for super-heavy artillery systems. Range
modifies accuracy and damage in important ways, something which will be
discussed in the damage subtypes section.

\begin{center}
    \begin{tabular}{ | l | l | }
    \hline
    Range & Weapon Type \\ \hline
    455--980 m & Infantry small arms. \\
    1225 m & Grenade launchers.\\
    525--1400 m & Infantry launchers. \\
    1575--1750 m & Autocannons. \\    
    1575--2275 m & Tank cannons. \\    
    1050--2975 m & ATGMs. \\
    2975--3500 m & Air AGMs. \\
    4200--5250 m & SEAD missiles. \\
    Up to 6125 m & Naval guns. \\
    3850--7700 m & Mortars. \\
    12200--42000 m & Artillery. \\ \hline
    \end{tabular}
\end{center}

Ships are considered part of the ground domain, and any weapons that can fire
onto ground units can fire onto ships. Whether or not these weapons are
effective are another question.

\begin{center}
    \begin{tabular}{ | l | l | }
    \hline
    Range & ASM Type \\ \hline
    4200--4900 m & Helicopter ASMs. \\
    4200--6300 m & Air ASMs. \\
    4900--7700 m & Ground AShMs. \\
    5250--9450 m & Naval ASMs. \\ \hline
    \end{tabular}
\end{center}

\subsection{Range -- Helicopters}

Range against helicopters. Autocannons and machine guns can attack helicopters
are close range. On planes, infrared (short) AAMs can attack planes, as can
their cannons.

Note that guided anti-air missiles that fire at helicopters that encroach their
range will fail if the helicopter then leaves their range while the missile is
in the air.

\begin{center}
    \begin{tabular}{ | l | p{4cm} | }
    \hline
    Range & Weapon Type \\ \hline
    525 m & Grenade launchers.\\
    525--1050 m & Machine guns.\\
    1575 m & Autocannons. \\
    2100--2625 m & MANPADS systems. \\
    1750--2800 m & Anti-air guns. \\
    2275--3325 m & IR anti-air missiles. \\
    2100--3500 m & Radar anti-air missiles. \\
    Up to 3500 m & Naval anti-air missiles. \\
    \hline
    \end{tabular}
\end{center}

\begin{center}
    \begin{tabular}{ | l | p{5cm} | }
    \hline
    Range & Weapon Type \\ \hline
    1575 m & Unguided air-to-air rockets.\\
    1575 m & Autocannons.\\
    1750 m & Short AAMs.\\    
    \hline
    \end{tabular}
\end{center}

\subsection{Range -- Airplanes}

Range against airplanes. Most weapons cannot attack airplanes---defense\\
against them is the domain of dedicated anti-air units.

Note that guided anti-air missiles that fire at airplanes that encroach their
range will not fail if the airplane leaves their range while the missile is in
the air.

\begin{center}
    \begin{tabular}{ | l | p{4.5cm} | }
    \hline
    Range & Weapon Type \\ \hline
    1750--2625 m & Anti-air guns.\\
    1820--2625 m & IR anti-air missiles.\\
    3150--4550 m & Radar anti-air missiles. \\
    Up to 4900 m & Naval anti-air missiles. \\
    5600 m & PATRIOT air nukes. \\
    \hline
    \end{tabular}
\end{center}

\begin{center}
    \begin{tabular}{ | l | p{4.5cm} | }
    \hline
    Range & Weapon Type \\ \hline
    2100 m & Unguided air-to-air rockets.\\
    1575--2800 m & Autocannons.\\
    3150--4200 m & Short AAMs. \\
    4900--7700 m & Medium AAMs. \\
    10500--11900 m & Long AAMs. \\
    \hline
    \end{tabular}
\end{center}

\section{Attack Power}

\subsection{Caliber}

This statistic does not in of itself have any impact on the game, though
larger-caliber weapons generally do more damage.

\subsection{Ammunition Carried}

The number of rounds of ammunition carried for the weapon. This statistic
is most important for missile systems, which tend to come in limited numbers.
In most other cases you'll run out of fuel or health before you run out of
ammo. Since MLRS systems fire all of their rounds at once, their ammunition
carried acts as their salvo size. Ammunition ranges from 1 (some bomber
payloads) to 9000 (the STRV 103C's machine gun) rounds.

\begin{center}
    \begin{tabular}{ | l | l | }
    \hline
    Magazine Size & Weapon System \\ \hline
    2--8 missiles & Heavy anti-air systems. \\
    4--10 missiles & Medium anti-air systems. \\ 
    4--12 missiles & Light (IR) anti-air systems. \\
    2--12 rockets & Infantry launchers. \\
    20 or 50 rounds & Infantry sniper rifles. \\
    24--48 rounds & Tank cannon rounds. \\ 
    2--80 missiles & MLRS systems. \\
    10--96 rockets & Unguided rockets. \\
    330--840 rounds & Autocannons. \\
    330--2000 rounds & Anti-air guns. \\
    320--4800 rounds & Infantry small arms. \\
    300--9000 rounds & Mounted machine-guns. \\ \hline
    \end{tabular}
\end{center}

\subsection{AP Power}

AP stands for ``Armor Penetration'' and is a measure of the units
armor-penetrating power. AP ranges from ``None'' for HE-only weapons up to 30
for high-end AGMs, and all the way up to 200 for high-end AShMs.

\begin{center}
    \begin{tabular}{ | l | l | }
    \hline
    AP & Weapon System \\ \hline
    1--3 & Autocannons. \\
    5--10 & Cluster bombs. \\
    10--24 & Infantry anti-tank launchers. \\
    6--24 & Tank cannons. \\
    13--25 & ATGMs. \\
    26--30 & AGMs. \\
    60--200 & ASMs and AShMs. \\
    \hline
    \end{tabular}
\end{center}

The formula for AP damage differs depending on whether the system is KE,
kinetic, or HEAT, high-explosive anti tank. See those respective sections for a
breakdown.

\subsection{HE Power}

HE stands for ``High Explosive''. HE damage is directed against units without
armor, and against infantry in particular---weapons lacking AP will do their HE
damage instead. HE ranges from 0.5 for some light machine guns all the way up to
20 for 1000 kg bombs (and, in its hardy spirit of break design philosophies, 30
for the B-5's 2000 kg bomb). All weapons have either AP power or HE power;
weapons with only AP power (for instance, AGMs) cannot target infantry, which
can only be hurt by HE damage.

\begin{center}
    \begin{tabular}{ | l | l | }
    \hline
    HE & Weapon System \\ \hline
    0.5 & Flamethrowers. \\
    0.5--1 & Small Arms. \\
    1 & Autocannons, grenade launchers.\\
    3--4 & Sapery launchers. \\
    3--4 & Tank cannons.\\
    3--8 & Air-to-air missiles. \\
    3--9 & Anti-air missiles. \\
    10--20 & HE Bombs. \\
    30 & B-5. \\
    \hline
    \end{tabular}
\end{center}

Infantry small arms will only do damage at units with 0 (x1.0) or 1 (x0.1)
armor; units with higher armor, they cannot damage with their rifles. The damage
done by other HE weapons drops less precipitously---indeed, it does full damage
at armor 1---but still drops to 0.1 by armor 6, and to 0.01 by armor 14. For a
full list of HE versus armor values see
\url{https://www.dropbox.com/s/l2u8w7tuj7igiul/WargameRD_Hidden_Knowledge_Spreadsheet.xls}.

The most interesting application of this mechanic is that it controls how much
damage HE bombs do against armored units, for instance, how much damage two
precision-guided 20-HE Paveway II can be expected to do against top armor,
after the Nighthawk's recent buff.

\begin{center}
    \begin{tabular}{ | l | l | l | }
    \hline
    Damage & Roof Armor & Units \\ \hline
    40 & 0 & Unarmored units.\\
    40 & 1 & Bulletproofed units. Cavalry tanks.\\
    16  & 2 & Mid-low tiered tanks. A few suport units.\\
    12  & 3 & Mid-heavy tiered tanks.\\
    8  & 4 & Heavy and superheavy tanks.\\
    6  & 5 & T-80BU superheavy tank.\\
    \hline
    \end{tabular}
\end{center}

Take note of that the next time you try to Nighthawk a tank!

\subsection{HEAT}

Weapons with the HEAT (``High Explosive Anti Tank'') tag always do at least one
damage, even when their AP is less then the armor of their target, and they will
always fire if a target presents itself in range. All anti-tank missiles and
some light tank cannons are HEAT weapons.

For every unit of armor on the target less than the AP of the weapon, a HEAT
damage source will do 0.5 more damage. So for instance, two 30 AP missiles
attacking the 23 AP Challenger 2: $30-23=7$, $7\times 0.5=3.5$, $3.5 + 1 = 4.5$,
and and in the end, two 30-AP missiles leave a Challenger that's barely alive,
but alive nonetheless. No other land unit gets that honor.

\subsection{KE}

Weapons with the KE (``Kinetic Energy'') tag will not be able to do any damage
if their damage output is less then the target's directed armor---instead they
will display that their weapon is ``Inefficient''. However, every 175 m closer
to the target from maximum range that the unit gets, its AP rises by 1. This
allows weak tanks to seriously damage heavy ones at close ranges, and the BMPT
to clean up weak tanks at close ranges with its autocannon, for instance. This
is an important and desirable property for tanks to have---and indeed, the vast
majority do. Combined with the range bonus, which also occurs every 175 m under
maximum range, this makes maximum range an important aspect of damage output for
these units.

The formula for KE damage against units with at least one unit of armor on the
side attacked is ((AP - Armor)/2) + 1. So a 30 AP AGM does 5 points of damage to
a tank with 22 armor---the Challenger 2 is thus unique in that it is the only
tank, with its 23 armor, to survive two hits from 30 AP AGMs to the front.

The highest achievable gun AP in the game is a shot fired at point blank range
by an M1A2. 24 base damage plus a 12 AP bonus makes for 36 bonus damage per
shot, enough to one-shot any other vehicle with 17 or less armor.

Against vehicles with zero armor weapons with AP will instead do exactly
double their AP in damage. Factually this means that even the lowest-AP weapon
will kill unarmored targets in one shot (as, excepting autocannons, they all
deal more than the 5 AP required to do so), and is the larger half of the reason
that bulletproofing on units is a useful property.

For a tabular breakdown of AP damage see:
\url{https://www.dropbox.com/s/l2u8w7tuj7igiul/WargameRD_Hidden_Knowledge_Spreadsheet.xls}.

An important note to make is that cluster bombs do more damage then they may at
first seem to: they deal damage against units' top armor, which is always much
less then their ground-combat frontal armor.

For a set of armor damage efficiency visualizations, see
\url{http://imgur.com/a/TJCI7}.

\subsection{Firemodes}

\textbf{Fire and Forget} missiles, once launched, will behave independently of
the launching platform. After a recent engine tweak, they also behave
independently of whether or not the unit is still sighted after the missiles
have been launched---you can kill units you don't see quite often.

\textbf{Semi-Active} missiles must be guided to the target by their launching
platform, though the launcher need not stay still while the missile is in the
air. Targets that escape visual sight or exit range while the missile in
the air will be spared a hit chance. Helicopter-launched AGMs are a
notable exception (as are air ones), as they will continue to home while the
target is in sight, regardless of the launch platform's range to target. This
was a fairly recent buff.

\textbf{Guided} missiles perform similarly to SemAct ones except in that they
cannot be used on the move---the launch platform must stand still to aim, fire,
and direct the missile, and cannot move until after an impact or a miss.

Gunned weapons lack a firemode tag. Their to-hit calculations are made at
firing, and once the round is away it will hit or miss regardless of the states
of the target or of the launch platform.

\section{Accuracy}
\subsection{Base Accuracy}

Accuracy determines, all other variables notwithstanding, how often a weapon
will hit a target. Base accuracy ranges from 10 percent to 75 percent, with the
exception of some ASM launch platforms, which can reach up to 85 percent
accuracy (ships are\ldots kind of hard to miss). A weapon's damage output
depends on its accuracy and its rate of fire: thus while 25\% accuracy is all
right for a Vulcan firing at 122 RPM, 30\% is unusable for a MiG-21PFM loaded
with high-value AGMs.

\begin{center}
    \begin{tabular}{ | l | l | }
    \hline
    Accuracy & Weapon Type \\ \hline
    10--30 \% & Machine guns.\\
    15--70 \% & Tank cannons, autocannons.\\
    25--65 \% & Guided missiles.\\
    35--70 \% & ATGMs.\\
    40--85 \% & ASMs. \\
    \hline
    \end{tabular}
\end{center}

\subsection{Size}

A unit's size applies a relatively small accuracy buff or debuff for units
firing on it. Sizes range from ``Very Small'' to ``Big'', and this stat occurs
in all units besides ships and planes, including infantry, which are
universially Very Small. The magnitude of the effect ranges from -10\% to
+5\%. The cutoffs for what is considered ``medium'' or ``large'' is very
arbitrary.

\begin{center}
    \begin{tabular}{ | l | p{3cm} | }
    \hline
    Size & Accuracy Change \\ \hline
    Very Small & -10\% \\
    Small & -5\% \\ 
    Medium & No change. \\ 
    Large & +5\% \\ \hline
    \end{tabular}
\end{center}

\subsection{ECM}

ECM is a statistic present on aircraft that does what Size does for ground
units, but to a much larger extent, causing incoming missile (and gun) fire to
be progressively less accurate than normal. ECM ranges between 0\% and 60\%,
where only the EF-111A Raven has 60\% ECM---the rest cap out at 50\%.

\begin{center}
    \begin{tabular}{ | l | p{3cm} | }
    \hline
    ALB Descriptor & Percentage Debuff \\ \hline
    None & 0\% \\
    Bad & -10\% \\ 
    Medium & -20\% \\ 
    Good & -30\% \\ 
    Very Good & -40\% \\
    Exceptional & -50\% \\
    Exceptional & -60\% \\
    \hline
    \end{tabular}
\end{center}

\subsection{Distance}

Every 175 m closer to a target a unit is below maximum range increases its
accuracy by 5\%. Thus, at point blank range, even tank cannons which are
noramlly 20\% accurate hit almost every shot. The maximum achievable accuracy is
still 85\%, as always, meaning that accurate weapons, at close range, have a
very high chance of causing a critical hit. This also means that a higher
maximum range tends to contribute to a units accuracy in combat.

This effect is present only on direct-fire weaponry---guns, autocannons,\\
grenade launchers, and so on. Missiles do not get an accuracy bonus no matter
what their distance to the target is---thus, ATGM teams tend to fare worse
against target at point blank ranges than high-quality anti-tank launchers.

To get a weapon's maximum accuracy bonus, divide the weapon's range by 175 and
then round down if you get an integer, or subtract one if you get a whole number
(you have to do this because the accuracy bonus is incremental, and because
units cannot ever be within zero meters of one another).

For example: the gun with the longest range in the game is the BMP-3's 2450 m
range 2A70 cannon (all other cannons max out at 2275 m). 2450 / 175 = 14, and we
subtract 1 from this to get 13. Multiply by 5\% to get the game-maximum +65\%
accuracy bonus for firing at point blank range on land.

The maximum air-to-air bonus is +75\% on 2800 m range air-to-air autocannons.

The maximum naval bonus and the maximum bonus in the game is a whopping +170\%
on the Kongo's 6125 m range main cannon.

An interesting property to make note of: most, though not all, of the weapon
ranges in Wargame are multiples of 175 m.

\subsection{Veterancy Bonus to Accuracy}

Probably the most important effect of veterancy is the bonus it gives to
accuracy. These bonuses occur in multiples of 8\%.

\begin{center}
    \begin{tabular}{ | l | l | }
    \hline
    Veterancy & Bonus \\ \hline
    Rookie & 0\% \\
    Trained & +8\% \\ 
    Hardened & +16\% \\ 
    Veteran & +24\% \\ 
    Elite & +32\% \\
    \hline
    \end{tabular}
\end{center}

\subsection{Critical Hits}

There is a base 1\% chance that whenever a unit is hit by a weapon it will
recieve a critical hit with an additional, randomized, negative impact. Even
weapons that fail to damage the unit can impart critical hits, and, naturally,
the higher the weapons rate of fire, the larger the chance and incidence of
critical hits it imparts. Critical hits can be inflicted on any unit---vehicle,
tank, plane, helicopter, or ship---except infantry. Since critical hits tend to
cause additional damage as part and parcel of their impact, critical hits caused
by sufficiently high-damage units can insta-kill where it normally wouldn't be
possible, or cause damage with weapons that normally could not do any. Heavy
anti-air missile carriers generally, and BUK-M1s specifically, are famous for
this.

At the same time, the maximum accuracy in the game, accounting for all bonuses,
is 85\%. What links critical hits and accuracy together is the fact that any
accuracy achieved that goes above 85\% is added onto the critical hit chance
instead.

For instance, while tank cannons rarely deal criticals at extended ranges, they
can impart life-threatening ones at point-blank ranges, a fact that
particularly works in the favor of weaker-gunned tanks.

The highest possible critical hit chance on the ground is that of super-heavy
tank firing its main gun at point-blank range. With 70\% base accuracy, a
+32\% Elite veterancy bonus, a +60\% range bonus, and for kicks a +5\% size
bonus, this translates to a 70\% critical hit chance.

The highest critical hit in naval battles is that of a Kongo
firing its main gun at point blank range. Kongos come in Hardened (+16\%) with
a cannon with 40\% base accuracy, and its maximum range bonus is +170\%. This
translates to 85\% accuracy and 40.28\% critical hit chance at point blank
range---indeed, ships engaging in naval battles at point blank range quickly
get covered in critical hit effects.

\subsection{Stabilizers}

Stabilizers, only present on guns and gunned units, are a measure of how
accurate a weapon is while firing on the move. No weapon is as accurate firing
on the move as it would be while standing still, but some can still maintain a
decent rate of damage while maneuvering. Missiles do not have stabalizers, for
reasons that should be obvious. Stabilizers used to be applied as a multiple to
overall accuracy, but in Red Dragon this was changed to flat hit chance while
moving. This statistic range from 0\% or ``None'', in which case the unit cannot
fire on the move at all, to 65\% for the highest-quality heavy tank stabalizers.

\subsection{Effect of Morale on Accuracy}

A unit's morale has a strong effect on accuracy (the precise mechanics of
morale are the topic of another section). The effects of morale are -20\%
accuracy for a worried unit, -40\% for a shaken unit, and -60\% for a panicked
one. The minimum accuracy in the game is a hilariously impotent 4\% for
rookie innaccurate panicked machine gun units firing at the edge of their range.

\section{Turrets}

All weapons in the game are mounted on ``turrets'' in engine terminology, even
when this isn't strictly the case (for instance, missile carriers).

Different turrets have different cones of fire. Some have a 0 degree cone of
fire---they must be aimed exactly at the target in order to fire (for instance,
tank cannons).

For these units the most important attribute is the unit's and the turret's turn
rate---how quickly the unit can pivot itself and/or its weapon around to take
another shot. This is an important attribute even with weapons with a more
generous cone of fire (particularly airplanes) as it determines the speed with
which they can turn into an attack.

Cannons have a 0 degree cone of fire, for obvious reasons. Most
missiles---short AAMs, medium AAMs, AGMs, AShMs---have a 70 degree cone of
fire, but there is a great deal of variation amongst naval
ASMs. Long AAMs have a 30 degree cone of fire. Ground-based anti-air missiles
have a 360 degree of fire---they target independently of where their turret is
facing.

The B-5 once again breaks the rules, this time in the company of the IL-102, by
mounting a backwards-facing turret that can only fire at units behind the plane.

\section{Rate of Fire}

Rate of fire tells you how quickly a weapon fires in either rounds per minute
or in terms of reload time.

The rate of fire displayed in unit cards is, in many cases, a highly derived
statistic that makes little sense. This is particularly true of fast-firing
machine guns and cannons, and can even affect missile carriers---the Gazelle
341F Celtic, for instance, lists a 20 second reload time for some reason. Rate
of fire tries to take into account a lot of in-game variables, and
basically, completely fails to do so.

\subsection{Aim Time}

There is a certain aim time that takes place before every burst,
shell, or missile, and begins as soon as the target enters sight. This does not
account for whether or not the turret is turned to face the target---a tank
that turns around to hit a faraway missile carrier will often manage an
``instant'' shot because turning it turret clocked its entire aim
time.

\begin{center}
    \begin{tabular}{ | l | l | }
    \hline
    Aim Time & Weapon Type \\ \hline
    0.2--1 sec & Anti-air guns. \\
    0.4 sec & Most weapons. \\
    2 sec & Interceptor missiles. \\ 
    4--10 sec & Mortars. \\
    10 sec & MLRS systems. \\
    10--35 sec & Artillery. \\
    \hline
    \end{tabular}
\end{center}

There are exceptions. Interceptor missiles have a 2 second aim time.

If a unit exits vision or range, even briefly, the unit's aiming is disrupted.
This causes missiles to auto-miss and projectile weapons currently being aimed
to have to restart aiming.

After the shot is fired, or lands or misses in the case of guided weapons, the
unit must re-aim its weapon. taking at a minimum another 0.4 seconds to fire
again.

\subsection{Reload Time}

All units have a reload time. Weapons begin their next reload cycle as soon as
the current weapon is fired. This allows missiles fired at long ranges, for
instance, to reload almost instantly. Note that since weapons still have to
re-aim after shots you still have to wait a minimum of 0.4 seconds to fire
again, but if the weapon has not finished reloading yet, you will have to wait
longer.

\subsection{Effect of Morale on Aim and Reload Time}

Morale damage degrades aim time and reload time. Units take 33\% longer to aim
and reload when worried, 100\% longer when shaken, and 200\% longer when
panicked.

\subsection{Autoloaders}

However, units with autoloaders, something not explicitly shown in the game
but which will soon be added into the game as an AUTO tag, will not take
longer to reload their weapons---though they will still take longer to aim.

See \url{http://www.wargame-ee.com/forum/viewtopic.php?f=155&t=47156} for a list
of units with autoloaders.

\subsection{Bursts}

All weapons in the game are fired in bursts, even if they are only
bursts of one in the case of missiles and tank cannons. Each burst has an
equivalent rate of fire and equivalent burst fire time; at the end of the burst
the unit must reload according to its reload time.

The ratio of burst time to reload time is what determines a unit's rate of fire.

For an analysis of AAA rates of fire see
\url{http://www.wargame-ee.com/forum/viewtopic.php?f=155&t=47156}.

The mechanics of burst fire is the primary reason why the ROF statistic in the
armory can be so wildly inarticulate.

Probably the most visible instance of burst fire mechanics is the one-two firing
of MiG-31, 31M, and F-14 Tomcat long AAMs, with a one second delay between
individual missiles.

\subsection{Effect of Squad Size}

The size of an infantry squad affects its combat ability, but not in the way you
would expect. As the size of a squad decreases, whether due to squad size or to
damage the squad has recieved, the unit's reload time increases. This is
due to the way that the engine works---burst length and rate of fire are
immutable, as infantry units are considered a special type of vehicle by the
engine.

This effect compounds with morale, making heavily damaged and suppressed
infantry units extremely slow to fire (and extremely inaccurate at that).

\section{Morale}

\subsection{Suppression}

All weapons in the game deal suppression damage, anywhere between 46 and 1500
suppression damage per impact. Weapons will deal their full suppression
in damage on imapct, but near misses will also do damage within a certain
``suppression radius'' that is related to the weapon's dispersion
radius.

All units enter the game with 0 suppression damage, and the maximum damage a
unit can have at any one time is 800.

\begin{center}
    \begin{tabular}{ | l | l | }
    \hline
    Suppression & Morale \\ \hline
    0 & Fine. \\
    200 & Worried. \\
    400 & Shaken. \\
    600 & Panicked. \\
    800 & Suppression cealing. \\
    \hline
    \end{tabular}
\end{center}

Armor has nothing to do with suppression, and for this reason even weapons which
do no or minimal damage to a target will auto-fire at 350 m or less for
the purposes of suppressing the target.

\subsection{Suppression Cooldown}

Units which have recieved suppression damage slowly recover their morale. The
base morale recovery rate is 20 morale per second, and it is increased by
veterancy.

\begin{center}
    \begin{tabular}{ | l | c | l | }
    \hline
    Bonus & Rate & Veterancy \\ \hline
    +0\% & 20 & Rookie. \\
    +150\% & 30 & Trained. \\
    +200\% & 40 & Hardened. \\
    +250\% & 50 & Veteran. \\
    +300\% & 60 & Elite. \\
    \hline
    \end{tabular}
\end{center}

A Rookie unit at 800 suppression will take 40 seconds to return to Calm, while
an Elite one in the same position will take only 13 1/3 seconds.

\subsection{Stunning}

If a unit takes 300 suppression damage within the space of a second it will be
stunned for 8 seconds, rendering it unable to do anything and breaking whatever
action it was taking before it got stunned (helicopters can still move while stunned, but
will flip around in the air perhaps unrealistically). Not currently know how
long a stun lasts, but units do seem to be immune to stunning\ldots while
stunned. They temporarily do not recover morale, however. A unit that gets out
of a stun has to repeat its aiming cycle, though not its reload.

Some say that airplanes that are stunned lose ECM protection. This has not been
tested, but does seem likely.

\subsection{Veterancy Bonus against Stun Time}

Stuns are much more effective against lower-veterancy units then against more
elite ones.

\begin{center}
    \begin{tabular}{ | l | l | l |}
    \hline
    Bonus & Stun Time & Veterancy \\ \hline
    -0\%  & 8 sec. &  Rookie. \\
    -19\% & 6.4 sec. & Trained. \\
    -39\% & 4.8 sec. & Hardened. \\
    -60\% & 3.2 sec. & Veteran. \\
    -80\% & 1.6 sec. & Elite. \\
    \hline
    \end{tabular}
\end{center}

\section{Dispersion}

\subsection{Artillery and Mortar Dispersion}

Dispersion is a statistic that's listed up-front on artillery and mortars. The
lower the dispersion, the tigher the artillery's cone of fire---the more
accurate it is.

Artillery has the same dispersion at all ranges in which it can fire. True
dispersion in the game is actually the listed dispersion divided by
74.28571429\ldots . In-game dispersion ranges between 3640 and 9400 m, or
between approximately 49 and 127 m.

Mortars have between 2275 and 3640 m of dispersion, which translates to about 30
to 49 m of dispersion.

Artillery and mortars have half as much dispersion against locations that are
spotted by other units.

\subsection{MLRS Dispersion}
MLRS dispersion is the same as artillery dispersion, except for the fact that
MLRS systems have a minimum and maximum dispersion value---they are
significantly more accurate at close ranges than at long ones.

MLRS dispersion ranges wildly between 35 m minimum for the magical ATACAMS
cluster rocket MLRS, and 2624 m for a BM-30 Smerch firing at the edge of its
range.

\subsection{Dispersion for non indirect-fire weapons}
For non-artillery units, dispersion is important because it interplays with\\
suppression---even when a weapon misses, if it's not a missile it can still
cause suppression damage. You can see the dispersion of a weapon manually by
telling it to fire pos at something---the circle that is drawn is the radius of
your shot, and therefore, your dispersion. Highly accurate weapons can also be
used to fire pos against targets that cannot be seen, but can be splashed.

Non indirect-fire units do not list a dispersion in their statistics, but you
can infer it from their accuracy, as there is a scalar relation.

\subsection{Veterancy Bonus against Dispersion}

Veterancy decreases dispersion. However, according to the armory tooltip this
only applies to artillery units.

\begin{center}
    \begin{tabular}{ | l | l | l |}
    \hline
    Bonus & Veterancy \\ \hline
    -0\%  &  Rookie. \\
    -10\% & Trained. \\
    -19\% & Hardened. \\
    -30\% & Veteran. \\
    -39\% & Elite. \\
    \hline
    \end{tabular}
\end{center}

\subsection{Morale Malus to Dispersion}

Unknown.

\section{Ballistic Velocity}

Different weapons in Wargame have different airspeeds. This is most important
for missiles, which have a relatively low airspeed---the faster the missile the
less of a chance your foe has of getting out of ranging and escaping, something
particularly important to helicopters.

Small-caliber machine guns are hitscan weapons---their bullets instantly reach
and damage the opponent, assuming they land the shot. This is done to
(greatly) reduce hardware stress, since bullets reach their target almost
instantly anyway.

Most missiles seem to have one of two airspeeds, a slow speed used by most
ground-attack missiles (TOWs and down, Kokons and down), ground or naval
ASMs, and heavy anti-air missiles, and a fast speed used for high-end
ground-attack missiles (Hellfire, Ataka V, Vikhr, most REDFOR tank ATGMs), air
AGMs, air ASMs, SEAD missiles, and light anti-air missiles.

All cannons and autocannons seem to fire at the same high, but not instantenous,
velocity.

Testing shows that the speeds of the various air-to-air missiles are,
approximately, 2500 meters per second for long AAMs; 2400 meters per second for
medium AAMs; and 2100 meters per second for short AAMs.

\section{Supply}

Supply units contain a certain amount of supply, which are provided to units
that have not changed morale status or been delt damage for a sufficient
length of time (uncertain), in a certain radius. Multiple units can draw supply
at once---there is no upper limit. There are three types of supply, fuel supplies, repair
supplies, and ammunition supplies, and units can draw all three simultaneously.

Note that, like, health, supply carried is stored as a float, but displayed as
an integer.

\subsection{Supply Rates}

Supply works the same for fuel and health regardless of the unit. However, the
supply rate for different weapons varies as it is dependent on their supply
cost---an arbitrary number that determines how much supply a unit of ammunition
of the weapon consumes.

\begin{center}
    \begin{tabular}{ | l | c | c | c | }
    \hline
    Type & Per Second Cost & Per Second Supplied & Per Unit Supplied Cost \\
    \hline
    	Fuel & 5 & 30 & 1/6 per unit of fuel.\\
    	Health & 5 & 0.1 & 50 per unit of health.\\
    	Ammo & 25 & 25 & Varies by weapon.\\
    \hline
    \end{tabular}
\end{center}

\subsection{Ammo Supply Rates}

A unit of ammo takes anywhere between 5 and 3000 units of supply to resupply.
This means that the maximum resupply rate is 5 units of ammunition per second,
and the minimum is two minutes.

\subsection{Air Unit Resupply}

Air units automatically resupply in the airport once they have evacuated. If
they have to resupply any of their individual weapon loadouts, they will have to
resupply\emph{all} rounds of the weapons in that category, which can have quite
a large effect on how long you have to wait before your unit is operationally
available again.

%\section{Derived Statistics}

%These statistics aren't listed, but can be derived from other that are, or
% from the game files.

%\subsection{Weapon Firing Cycle}
%
%An easy calculation: the ASU-85M carries 45 cannon rounds fired individually
% and has a rate of fire of 8 RPM. This means it takes it 5.625 minutes or 337.5
%seconds to fire all of its rounds, with 7.5 second pauses between weapon
%firings.

%A harder calculation: the ASU-85M also carries 500 DShk MG rounds. According to
%the unit's ``Salves'' attribute in the game files it fires five salvos, so it
%must fire 100 rounds per salvo. With a rate of fire of 500 RPM, this translates
%to

%\subsection{Fuel Efficiency}

%\subsection{Critical Hit Rate}

%\subsection{Instant Kill Chance}

%\subsection{Suppression Rate}

\end{document}
